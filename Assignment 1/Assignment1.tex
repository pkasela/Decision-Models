\documentclass[]{article}
\usepackage{lmodern}
\usepackage{amssymb,amsmath}
\usepackage{ifxetex,ifluatex}
\usepackage{fixltx2e} % provides \textsubscript
\ifnum 0\ifxetex 1\fi\ifluatex 1\fi=0 % if pdftex
  \usepackage[T1]{fontenc}
  \usepackage[utf8]{inputenc}
\else % if luatex or xelatex
  \ifxetex
    \usepackage{mathspec}
  \else
    \usepackage{fontspec}
  \fi
  \defaultfontfeatures{Ligatures=TeX,Scale=MatchLowercase}
\fi
% use upquote if available, for straight quotes in verbatim environments
\IfFileExists{upquote.sty}{\usepackage{upquote}}{}
% use microtype if available
\IfFileExists{microtype.sty}{%
\usepackage{microtype}
\UseMicrotypeSet[protrusion]{basicmath} % disable protrusion for tt fonts
}{}
\usepackage[margin=1in]{geometry}
\usepackage{hyperref}
\hypersetup{unicode=true,
            pdftitle={Assignment 1: Printed circuits},
            pdfauthor={Pranav Kasela},
            pdfborder={0 0 0},
            breaklinks=true}
\urlstyle{same}  % don't use monospace font for urls
\usepackage{color}
\usepackage{fancyvrb}
\newcommand{\VerbBar}{|}
\newcommand{\VERB}{\Verb[commandchars=\\\{\}]}
\DefineVerbatimEnvironment{Highlighting}{Verbatim}{commandchars=\\\{\}}
% Add ',fontsize=\small' for more characters per line
\usepackage{framed}
\definecolor{shadecolor}{RGB}{248,248,248}
\newenvironment{Shaded}{\begin{snugshade}}{\end{snugshade}}
\newcommand{\KeywordTok}[1]{\textcolor[rgb]{0.13,0.29,0.53}{\textbf{#1}}}
\newcommand{\DataTypeTok}[1]{\textcolor[rgb]{0.13,0.29,0.53}{#1}}
\newcommand{\DecValTok}[1]{\textcolor[rgb]{0.00,0.00,0.81}{#1}}
\newcommand{\BaseNTok}[1]{\textcolor[rgb]{0.00,0.00,0.81}{#1}}
\newcommand{\FloatTok}[1]{\textcolor[rgb]{0.00,0.00,0.81}{#1}}
\newcommand{\ConstantTok}[1]{\textcolor[rgb]{0.00,0.00,0.00}{#1}}
\newcommand{\CharTok}[1]{\textcolor[rgb]{0.31,0.60,0.02}{#1}}
\newcommand{\SpecialCharTok}[1]{\textcolor[rgb]{0.00,0.00,0.00}{#1}}
\newcommand{\StringTok}[1]{\textcolor[rgb]{0.31,0.60,0.02}{#1}}
\newcommand{\VerbatimStringTok}[1]{\textcolor[rgb]{0.31,0.60,0.02}{#1}}
\newcommand{\SpecialStringTok}[1]{\textcolor[rgb]{0.31,0.60,0.02}{#1}}
\newcommand{\ImportTok}[1]{#1}
\newcommand{\CommentTok}[1]{\textcolor[rgb]{0.56,0.35,0.01}{\textit{#1}}}
\newcommand{\DocumentationTok}[1]{\textcolor[rgb]{0.56,0.35,0.01}{\textbf{\textit{#1}}}}
\newcommand{\AnnotationTok}[1]{\textcolor[rgb]{0.56,0.35,0.01}{\textbf{\textit{#1}}}}
\newcommand{\CommentVarTok}[1]{\textcolor[rgb]{0.56,0.35,0.01}{\textbf{\textit{#1}}}}
\newcommand{\OtherTok}[1]{\textcolor[rgb]{0.56,0.35,0.01}{#1}}
\newcommand{\FunctionTok}[1]{\textcolor[rgb]{0.00,0.00,0.00}{#1}}
\newcommand{\VariableTok}[1]{\textcolor[rgb]{0.00,0.00,0.00}{#1}}
\newcommand{\ControlFlowTok}[1]{\textcolor[rgb]{0.13,0.29,0.53}{\textbf{#1}}}
\newcommand{\OperatorTok}[1]{\textcolor[rgb]{0.81,0.36,0.00}{\textbf{#1}}}
\newcommand{\BuiltInTok}[1]{#1}
\newcommand{\ExtensionTok}[1]{#1}
\newcommand{\PreprocessorTok}[1]{\textcolor[rgb]{0.56,0.35,0.01}{\textit{#1}}}
\newcommand{\AttributeTok}[1]{\textcolor[rgb]{0.77,0.63,0.00}{#1}}
\newcommand{\RegionMarkerTok}[1]{#1}
\newcommand{\InformationTok}[1]{\textcolor[rgb]{0.56,0.35,0.01}{\textbf{\textit{#1}}}}
\newcommand{\WarningTok}[1]{\textcolor[rgb]{0.56,0.35,0.01}{\textbf{\textit{#1}}}}
\newcommand{\AlertTok}[1]{\textcolor[rgb]{0.94,0.16,0.16}{#1}}
\newcommand{\ErrorTok}[1]{\textcolor[rgb]{0.64,0.00,0.00}{\textbf{#1}}}
\newcommand{\NormalTok}[1]{#1}
\usepackage{graphicx,grffile}
\makeatletter
\def\maxwidth{\ifdim\Gin@nat@width>\linewidth\linewidth\else\Gin@nat@width\fi}
\def\maxheight{\ifdim\Gin@nat@height>\textheight\textheight\else\Gin@nat@height\fi}
\makeatother
% Scale images if necessary, so that they will not overflow the page
% margins by default, and it is still possible to overwrite the defaults
% using explicit options in \includegraphics[width, height, ...]{}
\setkeys{Gin}{width=\maxwidth,height=\maxheight,keepaspectratio}
\IfFileExists{parskip.sty}{%
\usepackage{parskip}
}{% else
\setlength{\parindent}{0pt}
\setlength{\parskip}{6pt plus 2pt minus 1pt}
}
\setlength{\emergencystretch}{3em}  % prevent overfull lines
\providecommand{\tightlist}{%
  \setlength{\itemsep}{0pt}\setlength{\parskip}{0pt}}
\setcounter{secnumdepth}{0}
% Redefines (sub)paragraphs to behave more like sections
\ifx\paragraph\undefined\else
\let\oldparagraph\paragraph
\renewcommand{\paragraph}[1]{\oldparagraph{#1}\mbox{}}
\fi
\ifx\subparagraph\undefined\else
\let\oldsubparagraph\subparagraph
\renewcommand{\subparagraph}[1]{\oldsubparagraph{#1}\mbox{}}
\fi

%%% Use protect on footnotes to avoid problems with footnotes in titles
\let\rmarkdownfootnote\footnote%
\def\footnote{\protect\rmarkdownfootnote}

%%% Change title format to be more compact
\usepackage{titling}

% Create subtitle command for use in maketitle
\providecommand{\subtitle}[1]{
  \posttitle{
    \begin{center}\large#1\end{center}
    }
}

\setlength{\droptitle}{-2em}

  \title{Assignment 1: Printed circuits}
    \pretitle{\vspace{\droptitle}\centering\huge}
  \posttitle{\par}
    \author{Pranav Kasela}
    \preauthor{\centering\large\emph}
  \postauthor{\par}
      \predate{\centering\large\emph}
  \postdate{\par}
    \date{18-03-2019}


\begin{document}
\maketitle

\subsection{Introduction}\label{introduction}

\emph{MC Manufacturing} has contracted to provide \emph{DISCO
Electronics} with printed circuit (``PC'') boards under the following
terms: (1) 100,000 PC boards will be delivered to DISCO in one month,
and (2) DISCO has an option to take delivery of an additional 100,000
boards in three months by giving Aba 30 days notice. DISCO will pay
\$5.00 for each board that it purchases. MC manufactures the PC boards
using a batch process, and manufacturing costs are as follows: (1) there
is a fixed setup cost of \$250,000 for any manufacturing batch run,
regardless of the size of the run, and (2) there is a marginal
manufacturing cost of \$2.00 per board regardless of the size of the
batch run. MC must decide whether to manufacture all 200,000 PC boards
now or whether to only manufacture 100,000 now and manufacture the other
100,000 boards only if DISCO exercises its option to buy those boards.
If MC manufactures 200,000 now and DISCO does not exercise its option,
then the manufacturing cost of the extra 100,000 boards will be totally
lost. MC believes there is a 50\% chance DISCO will exercise its option
to buy the additional 100,000 PC boards.

\subsection{The decision tree}\label{the-decision-tree}

\begin{itemize}
\tightlist
\item
  Draw a decision tree for the decision that MC faces.
\end{itemize}

\begin{Shaded}
\begin{Highlighting}[]
\KeywordTok{library}\NormalTok{(yaml)}
\KeywordTok{library}\NormalTok{(radiant)}
\KeywordTok{library}\NormalTok{(radiant.model)}
\end{Highlighting}
\end{Shaded}

\begin{Shaded}
\begin{Highlighting}[]
\KeywordTok{setwd}\NormalTok{(}\StringTok{"/home/pranav/Desktop/Decision Models Assignments/Assignment 1/"}\NormalTok{)}
\end{Highlighting}
\end{Shaded}

\begin{verbatim}
Error in setwd("/home/pranav/Desktop/Decision Models Assignments/Assignment 1/"): cannot change working directory
\end{verbatim}

\begin{Shaded}
\begin{Highlighting}[]
\NormalTok{tree =}\StringTok{ }\KeywordTok{yaml.load_file}\NormalTok{(}\DataTypeTok{input =} \StringTok{"./Board_Production.yaml"}\NormalTok{)}

\NormalTok{result =}\KeywordTok{dtree}\NormalTok{(}\DataTypeTok{yl =}\NormalTok{ tree)}

\KeywordTok{plot}\NormalTok{(result, }\DataTypeTok{final =} \OtherTok{FALSE}\NormalTok{)}
\end{Highlighting}
\end{Shaded}

\hypertarget{htmlwidget-202b6d9078879e5c77ad}{}

In the Decision Tree initially \emph{MC Manufacturing} need to choose
either to produce the whole batch of \(200,000\) pieces toghether or to
do it in two different moments, each time \emph{MC Manufacturing} starts
the batch production procedure as mentioned in the introduction \emph{MC
Manufacturing} pays \(\$250,000\), \emph{MC Manufacturing} does not know
if \emph{DISCO Electronics} will buy the second batch, but \emph{MC
Manufacturing} knows its the probability.

\subsection{Expected value}\label{expected-value}

\begin{itemize}
\tightlist
\item
  Determine the preferred course of action for MC assuming it uses
  expected profit as its decision criterion.
\end{itemize}

\begin{Shaded}
\begin{Highlighting}[]
\KeywordTok{plot}\NormalTok{(result, }\DataTypeTok{final =} \OtherTok{TRUE}\NormalTok{)}
\end{Highlighting}
\end{Shaded}

\hypertarget{htmlwidget-231d0239b9eca5a0eae4}{}

The preferred course of action is to produce the two batches toghether
since it has an expected value of \(\$100,000\) against \(\$75,000\) of
the other branch in which \emph{MC Manufacturing} produces the two batch
separately.

\subsection{Utility Function and Certainty
Equivalent}\label{utility-function-and-certainty-equivalent}

Assume that all the information still holds, except assume now that MC
has an exponential utility function with a risk tolerance of
\(\$100,000\).

\begin{Shaded}
\begin{Highlighting}[]
\NormalTok{utilityFunctionExp <-}\StringTok{ }\ControlFlowTok{function}\NormalTok{(X, R) \{}
\NormalTok{res <-}\StringTok{ }\DecValTok{1}\OperatorTok{-}\StringTok{ }\KeywordTok{exp}\NormalTok{(}\OperatorTok{-}\NormalTok{X}\OperatorTok{/}\NormalTok{R)}
\KeywordTok{return}\NormalTok{(res)}
\NormalTok{\}}

\NormalTok{CertEquivalent =}\StringTok{ }\ControlFlowTok{function}\NormalTok{(EU, R)\{}
\NormalTok{CE =}\StringTok{ }\OperatorTok{-}\NormalTok{R}\OperatorTok{*}\KeywordTok{ln}\NormalTok{(}\DecValTok{1}\OperatorTok{-}\NormalTok{EU)}
\KeywordTok{return}\NormalTok{(CE)}
\NormalTok{\}}

\NormalTok{CalcExpectedUtilityFunction =}\StringTok{ }\ControlFlowTok{function}\NormalTok{(profit, R)\{}
  \CommentTok{#------Branch 1----------#}
\NormalTok{  UF1 =}\StringTok{ }\KeywordTok{utilityFunctionExp}\NormalTok{(profit}\OperatorTok{$}\NormalTok{profitBranch1, R)}
\NormalTok{  EU1 =}\StringTok{ }\NormalTok{UF1[}\DecValTok{1}\NormalTok{]}\OperatorTok{*}\FloatTok{0.5} \OperatorTok{+}\StringTok{ }\NormalTok{UF1[}\DecValTok{2}\NormalTok{]}\OperatorTok{*}\FloatTok{0.5}

  \CommentTok{#------Branch 2----------#}
\NormalTok{  UF2 =}\StringTok{ }\KeywordTok{utilityFunctionExp}\NormalTok{(profit}\OperatorTok{$}\NormalTok{profitBranch2, R)}
\NormalTok{  EU2 =}\StringTok{ }\NormalTok{UF2[}\DecValTok{1}\NormalTok{]}\OperatorTok{*}\FloatTok{0.5} \OperatorTok{+}\StringTok{ }\NormalTok{UF2[}\DecValTok{2}\NormalTok{]}\OperatorTok{*}\FloatTok{0.5}
  
  \CommentTok{#----Return Final Result----#}
  \KeywordTok{return}\NormalTok{(}\KeywordTok{c}\NormalTok{(EU1, EU2))}
\NormalTok{\}}

\NormalTok{CalcBranchCE =}\StringTok{ }\ControlFlowTok{function}\NormalTok{(profit, R)\{}
\NormalTok{CE_vett =}\StringTok{ }\KeywordTok{CertEquivalent}\NormalTok{(}\KeywordTok{CalcExpectedUtilityFunction}\NormalTok{(profit, R), R)}
\KeywordTok{return}\NormalTok{(CE_vett)}
\NormalTok{\}}

\CommentTok{#Create a DataFrame with Profits per Branch}

\NormalTok{index <-}\StringTok{ }\DecValTok{1}\OperatorTok{:}\DecValTok{2}
\NormalTok{profitBranch1 <-}\StringTok{ }\KeywordTok{c}\NormalTok{(}\DecValTok{35}\NormalTok{,}\OperatorTok{-}\DecValTok{15}\NormalTok{)}
\NormalTok{profitBranch2 <-}\StringTok{ }\KeywordTok{c}\NormalTok{(}\DecValTok{10}\NormalTok{,}\DecValTok{5}\NormalTok{)}
\NormalTok{profit <-}\StringTok{ }\KeywordTok{data.frame}\NormalTok{(}\StringTok{"X"}\NormalTok{=index,}\StringTok{"profitBranch1"}\NormalTok{=profitBranch1,}\StringTok{"profitBranch2"}\NormalTok{=profitBranch2)}
\NormalTok{R=}\DecValTok{10}
\end{Highlighting}
\end{Shaded}

\begin{itemize}
\tightlist
\item
  Determine MC's preferred course of action.
\end{itemize}

\begin{Shaded}
\begin{Highlighting}[]
\NormalTok{CE <-}\StringTok{ }\KeywordTok{CalcBranchCE}\NormalTok{(profit,R)}

\NormalTok{CE_Branch1 <-}\StringTok{ }\NormalTok{CE[}\DecValTok{1}\NormalTok{]}
\NormalTok{CE_Branch2 <-}\StringTok{ }\NormalTok{CE[}\DecValTok{2}\NormalTok{]}
\KeywordTok{cat}\NormalTok{(}\KeywordTok{paste0}\NormalTok{(}\StringTok{'Certainty Equivalent of Branch 1:'}\NormalTok{,CE_Branch1),}\KeywordTok{paste0}\NormalTok{(}\StringTok{'Certainty Equivalent of Branch 2:'}\NormalTok{,CE_Branch2),}\DataTypeTok{sep=}\StringTok{'}\CharTok{\textbackslash{}n}\StringTok{'}\NormalTok{)}
\end{Highlighting}
\end{Shaded}

\begin{verbatim}
Certainty Equivalent of Branch 1:-8.13568167929173
Certainty Equivalent of Branch 2:7.19070196379839
\end{verbatim}

We can see that the preferred course is the Branch 2, in which we wait
before producing the other batch, which could have been expected since
the R is positive thus the \emph{MC Manufacturing} is risk averse, and
the Branch 2 have 0 probabilty of loss, while in the Branch 1 \emph{MC
Manufacturing} has \(50\%\) chance of failing.

\subsection{Modification of the
process}\label{modification-of-the-process}

For the decision in the preceding point, MC Manufacturing has created a
new option: it can conduct some research and development in an attempt
to lower the fixed setup cost associated with manufacturing a batch of
the PC boards. This research and development would not be completed in
time to influence the setup cost for the initial batch that DISCO has
ordered, but would be completed before the second batch would have to be
manufactured. The research and development will cost \(\$25,000\), and
there is a 0.4 probability that it will be successful. If it is
successful, then the fixed setup cost per batch will be reduced by
\(\$200,000\) to \(\$50,000\). If the research and development is not
successful, then there will be no reduction in the setup cost. There
will be no other benefits from the research and development besides the
potential reduction in setup cost for the DISCO reorder.

\begin{itemize}
\tightlist
\item
  Using expected profit as the decision criterion, determine whether MC
  should undertake the research and development.
\end{itemize}

\begin{Shaded}
\begin{Highlighting}[]
\NormalTok{tree_RnD =}\StringTok{ }\KeywordTok{yaml.load_file}\NormalTok{(}\DataTypeTok{input =} \StringTok{"./Board_Production_RnD.yaml"}\NormalTok{)}
\NormalTok{result_RnD =}\StringTok{ }\KeywordTok{dtree}\NormalTok{(}\DataTypeTok{yl =}\NormalTok{ tree_RnD)}

\KeywordTok{plot}\NormalTok{(result_RnD, }\DataTypeTok{final =} \OtherTok{TRUE}\NormalTok{)}
\end{Highlighting}
\end{Shaded}

\hypertarget{htmlwidget-f3042a54266a04061cd0}{}

The R\&D(Research and Developement) branch has an Expected Value of
\(\$90,000\) which is still lower than the Branch 1 of producing
everything toghether, so using the expected profit the preferred course
is still Branch 1, thus \emph{MC Manufacturing} should not undertake the
R\&D.

\subsection{Value of Information}\label{value-of-information}

Using expected profit as the decision criteria, determine the value of
learning for certain whether the research and development will be
successful before a decision has to be made about whether to initially
manufacture \(100,000\) or \(200,000\) PC boards.

\begin{Shaded}
\begin{Highlighting}[]
\NormalTok{tree_PI =}\StringTok{ }\KeywordTok{yaml.load_file}\NormalTok{(}\DataTypeTok{input =} \StringTok{"./Board_Production_PI.yaml"}\NormalTok{)}
\NormalTok{result_PI =}\StringTok{ }\KeywordTok{dtree}\NormalTok{(}\DataTypeTok{yl =}\NormalTok{ tree_PI)}


\KeywordTok{plot}\NormalTok{(result_PI, }\DataTypeTok{final =} \OtherTok{TRUE}\NormalTok{)}
\end{Highlighting}
\end{Shaded}

\hypertarget{htmlwidget-a47e42890b0384f79af3}{}

The Expected Value using this information is \(\$120,000\) while the
expected value without the information on the success of R\&D was
\(\$100,000\) so the value of information using the expected profit is
\(\$120,000 - \$100,000 = \$20,000\).


\end{document}
